% Options for packages loaded elsewhere
\PassOptionsToPackage{unicode}{hyperref}
\PassOptionsToPackage{hyphens}{url}
%
\documentclass[
]{article}
\usepackage{lmodern}
\usepackage{amssymb,amsmath}
\usepackage{ifxetex,ifluatex}
\ifnum 0\ifxetex 1\fi\ifluatex 1\fi=0 % if pdftex
  \usepackage[T1]{fontenc}
  \usepackage[utf8]{inputenc}
  \usepackage{textcomp} % provide euro and other symbols
\else % if luatex or xetex
  \usepackage{unicode-math}
  \defaultfontfeatures{Scale=MatchLowercase}
  \defaultfontfeatures[\rmfamily]{Ligatures=TeX,Scale=1}
\fi
% Use upquote if available, for straight quotes in verbatim environments
\IfFileExists{upquote.sty}{\usepackage{upquote}}{}
\IfFileExists{microtype.sty}{% use microtype if available
  \usepackage[]{microtype}
  \UseMicrotypeSet[protrusion]{basicmath} % disable protrusion for tt fonts
}{}
\makeatletter
\@ifundefined{KOMAClassName}{% if non-KOMA class
  \IfFileExists{parskip.sty}{%
    \usepackage{parskip}
  }{% else
    \setlength{\parindent}{0pt}
    \setlength{\parskip}{6pt plus 2pt minus 1pt}}
}{% if KOMA class
  \KOMAoptions{parskip=half}}
\makeatother
\usepackage{xcolor}
\IfFileExists{xurl.sty}{\usepackage{xurl}}{} % add URL line breaks if available
\IfFileExists{bookmark.sty}{\usepackage{bookmark}}{\usepackage{hyperref}}
\hypersetup{
  pdftitle={Re-evaluation of the Philips curve},
  pdfauthor={Dominik Gulácsy},
  hidelinks,
  pdfcreator={LaTeX via pandoc}}
\urlstyle{same} % disable monospaced font for URLs
\usepackage[margin=1in]{geometry}
\usepackage{longtable,booktabs}
% Correct order of tables after \paragraph or \subparagraph
\usepackage{etoolbox}
\makeatletter
\patchcmd\longtable{\par}{\if@noskipsec\mbox{}\fi\par}{}{}
\makeatother
% Allow footnotes in longtable head/foot
\IfFileExists{footnotehyper.sty}{\usepackage{footnotehyper}}{\usepackage{footnote}}
\makesavenoteenv{longtable}
\usepackage{graphicx,grffile}
\makeatletter
\def\maxwidth{\ifdim\Gin@nat@width>\linewidth\linewidth\else\Gin@nat@width\fi}
\def\maxheight{\ifdim\Gin@nat@height>\textheight\textheight\else\Gin@nat@height\fi}
\makeatother
% Scale images if necessary, so that they will not overflow the page
% margins by default, and it is still possible to overwrite the defaults
% using explicit options in \includegraphics[width, height, ...]{}
\setkeys{Gin}{width=\maxwidth,height=\maxheight,keepaspectratio}
% Set default figure placement to htbp
\makeatletter
\def\fps@figure{htbp}
\makeatother
\setlength{\emergencystretch}{3em} % prevent overfull lines
\providecommand{\tightlist}{%
  \setlength{\itemsep}{0pt}\setlength{\parskip}{0pt}}
\setcounter{secnumdepth}{-\maxdimen} % remove section numbering

\title{Re-evaluation of the Philips curve}
\usepackage{etoolbox}
\makeatletter
\providecommand{\subtitle}[1]{% add subtitle to \maketitle
  \apptocmd{\@title}{\par {\large #1 \par}}{}{}
}
\makeatother
\subtitle{Does higher inflation convey lower unemployment?}
\author{Dominik Gulácsy}
\date{02/01/2021}

\begin{document}
\maketitle
\begin{abstract}
This analysis is addressing the question whether the inverse
relationship between unemployment and inflation stated by the Philips
curve exists in modern economies. I use cross-sectional data on
countries from 2017. After running several regression models, my results
support that this inverse relationship does not hold for the present
(2016-2018). This finding has relevance to appropiately evaluate policy
decisions regarding unemployment and economic growth, and assess the
overall health of the economy.
\end{abstract}

\hypertarget{introduction}{%
\section{Introduction}\label{introduction}}

The main goal of this analysis is to test the Philips curve's validity
in current times. This economic theory was developed by
\href{https://onlinelibrary.wiley.com/doi/full/10.1111/j.1468-0335.1958.tb00003.x}{A.
W. Phillips} stating that inflation and unemployment have a stable and
inverse relationship. It was based on the idea that with economic growth
comes inflation, which in turn should lead to more jobs and less
unemployment. It sounds logical, however it was seriously called into
question during the times of stagflation in the 1970's.

The major difference in case of this analysis is that by conducting a
cross-sectional causal analysis I can get insight on whether the theory
stands in global terms instead of checking it for one country throughout
time. To eventually check the theory I run regressions by taking the
reciprocal of inflation as this should make the pattern of association
linear when there is indeed a an inverse relationship between
unemployment and inflation.

\hypertarget{data}{%
\section{Data}\label{data}}

The data used in the analysis is solely from the World Bank. I used the
WDI package to download the data in R. The data contains economic
indicators on 122 countries for 2017 after removing observations with
any missing values. I may add that it is highly probable that by this
smaller countries or countries that are in turmoil (e.g.: Venezuela)
will be underrepresented as they are more likely to have missing values.
Given that there are only few observations available I ignore this
issue.

Otherwise, the quality of the raw data is superior, it is
well-documented and provides good coverage. However, one may argue that
indicators are not truly comparable and may be politically biased as
they are from different national statistical agencies. Since there are
no readily available data to counter for these measurement errors I keep
in mind this as a limitation to results. Furthermore, to appropriately
check the inverse relationship I need to drop observations with
inflation below 0 since it messes up the linear regression. This likely
makes the sample even less representative as well. Finally, I end up
with 118 observations.

As I need to regress unemployment on inflation I want to compare
countries that can be considered similar in every other way but
inflation. Therefore I use some control variables in my analysis which
are the following: GDP Growth (\%), Savings (\% of GDP), Broad Money (\%
of GDP) and Government Expenditure (\% of GDP).

\begin{figure}
\centering
\includegraphics{unemployment_analysis_report_files/figure-latex/fig1-1.pdf}
\caption{Distribution of variables (2017)}
\end{figure}

As it can be seen on the histograms all variables are in relative terms.
This approach means that size is not relevant. Also, this makes
interpretation easier and variables more normal thus improving the fit
of the linear regression. Some variables are fairly normal while some
are closer to lognormal distribution (Broad Money, Inflation,
Unemployment). In case of money supply and unemployment it is likely due
to the fact that these measures cannot be negative while in case of
inflation it is the result of a technical requirement.

Taking the logarithm of these variables may alleviate non-normality but
interpretation would be hard so I decide not to take logs. Looking at
the distributions I can see some extreme values in gov. expenditure and
inflation. I investigate their effect as a robustness check later on.

\begin{longtable}[]{@{}lrrrrrr@{}}
\caption{Descriptive statistics of the variables (2017)}\tabularnewline
\toprule
\begin{minipage}[b]{0.07\columnwidth}\raggedright
statistics\strut
\end{minipage} & \begin{minipage}[b]{0.18\columnwidth}\raggedleft
Gov.~Expenditure (\% of GDP)\strut
\end{minipage} & \begin{minipage}[b]{0.10\columnwidth}\raggedleft
GDP Growth (\%)\strut
\end{minipage} & \begin{minipage}[b]{0.15\columnwidth}\raggedleft
Broad Money (\% of GDP)\strut
\end{minipage} & \begin{minipage}[b]{0.12\columnwidth}\raggedleft
Savings (\% of GDP)\strut
\end{minipage} & \begin{minipage}[b]{0.09\columnwidth}\raggedleft
Inflation (\%)\strut
\end{minipage} & \begin{minipage}[b]{0.11\columnwidth}\raggedleft
Unemployment (\%)\strut
\end{minipage}\tabularnewline
\midrule
\endfirsthead
\toprule
\begin{minipage}[b]{0.07\columnwidth}\raggedright
statistics\strut
\end{minipage} & \begin{minipage}[b]{0.18\columnwidth}\raggedleft
Gov.~Expenditure (\% of GDP)\strut
\end{minipage} & \begin{minipage}[b]{0.10\columnwidth}\raggedleft
GDP Growth (\%)\strut
\end{minipage} & \begin{minipage}[b]{0.15\columnwidth}\raggedleft
Broad Money (\% of GDP)\strut
\end{minipage} & \begin{minipage}[b]{0.12\columnwidth}\raggedleft
Savings (\% of GDP)\strut
\end{minipage} & \begin{minipage}[b]{0.09\columnwidth}\raggedleft
Inflation (\%)\strut
\end{minipage} & \begin{minipage}[b]{0.11\columnwidth}\raggedleft
Unemployment (\%)\strut
\end{minipage}\tabularnewline
\midrule
\endhead
\begin{minipage}[t]{0.07\columnwidth}\raggedright
mean\strut
\end{minipage} & \begin{minipage}[t]{0.18\columnwidth}\raggedleft
16.08\strut
\end{minipage} & \begin{minipage}[t]{0.10\columnwidth}\raggedleft
3.58\strut
\end{minipage} & \begin{minipage}[t]{0.15\columnwidth}\raggedleft
62.08\strut
\end{minipage} & \begin{minipage}[t]{0.12\columnwidth}\raggedleft
22.16\strut
\end{minipage} & \begin{minipage}[t]{0.09\columnwidth}\raggedleft
6.03\strut
\end{minipage} & \begin{minipage}[t]{0.11\columnwidth}\raggedleft
7.18\strut
\end{minipage}\tabularnewline
\begin{minipage}[t]{0.07\columnwidth}\raggedright
median\strut
\end{minipage} & \begin{minipage}[t]{0.18\columnwidth}\raggedleft
15.35\strut
\end{minipage} & \begin{minipage}[t]{0.10\columnwidth}\raggedleft
3.77\strut
\end{minipage} & \begin{minipage}[t]{0.15\columnwidth}\raggedleft
53.52\strut
\end{minipage} & \begin{minipage}[t]{0.12\columnwidth}\raggedleft
21.51\strut
\end{minipage} & \begin{minipage}[t]{0.09\columnwidth}\raggedleft
4.00\strut
\end{minipage} & \begin{minipage}[t]{0.11\columnwidth}\raggedleft
5.32\strut
\end{minipage}\tabularnewline
\begin{minipage}[t]{0.07\columnwidth}\raggedright
min\strut
\end{minipage} & \begin{minipage}[t]{0.18\columnwidth}\raggedleft
4.40\strut
\end{minipage} & \begin{minipage}[t]{0.10\columnwidth}\raggedleft
-4.71\strut
\end{minipage} & \begin{minipage}[t]{0.15\columnwidth}\raggedleft
12.67\strut
\end{minipage} & \begin{minipage}[t]{0.12\columnwidth}\raggedleft
-2.21\strut
\end{minipage} & \begin{minipage}[t]{0.09\columnwidth}\raggedleft
0.26\strut
\end{minipage} & \begin{minipage}[t]{0.11\columnwidth}\raggedleft
0.14\strut
\end{minipage}\tabularnewline
\begin{minipage}[t]{0.07\columnwidth}\raggedright
max\strut
\end{minipage} & \begin{minipage}[t]{0.18\columnwidth}\raggedleft
54.62\strut
\end{minipage} & \begin{minipage}[t]{0.10\columnwidth}\raggedleft
10.30\strut
\end{minipage} & \begin{minipage}[t]{0.15\columnwidth}\raggedleft
260.06\strut
\end{minipage} & \begin{minipage}[t]{0.12\columnwidth}\raggedleft
55.34\strut
\end{minipage} & \begin{minipage}[t]{0.09\columnwidth}\raggedleft
43.07\strut
\end{minipage} & \begin{minipage}[t]{0.11\columnwidth}\raggedleft
27.07\strut
\end{minipage}\tabularnewline
\begin{minipage}[t]{0.07\columnwidth}\raggedright
1st\_qu.\strut
\end{minipage} & \begin{minipage}[t]{0.18\columnwidth}\raggedleft
11.76\strut
\end{minipage} & \begin{minipage}[t]{0.10\columnwidth}\raggedleft
2.15\strut
\end{minipage} & \begin{minipage}[t]{0.15\columnwidth}\raggedleft
37.28\strut
\end{minipage} & \begin{minipage}[t]{0.12\columnwidth}\raggedleft
15.47\strut
\end{minipage} & \begin{minipage}[t]{0.09\columnwidth}\raggedleft
1.95\strut
\end{minipage} & \begin{minipage}[t]{0.11\columnwidth}\raggedleft
3.73\strut
\end{minipage}\tabularnewline
\begin{minipage}[t]{0.07\columnwidth}\raggedright
3rd\_qu\strut
\end{minipage} & \begin{minipage}[t]{0.18\columnwidth}\raggedleft
18.65\strut
\end{minipage} & \begin{minipage}[t]{0.10\columnwidth}\raggedleft
4.84\strut
\end{minipage} & \begin{minipage}[t]{0.15\columnwidth}\raggedleft
77.84\strut
\end{minipage} & \begin{minipage}[t]{0.12\columnwidth}\raggedleft
27.94\strut
\end{minipage} & \begin{minipage}[t]{0.09\columnwidth}\raggedleft
7.40\strut
\end{minipage} & \begin{minipage}[t]{0.11\columnwidth}\raggedleft
9.46\strut
\end{minipage}\tabularnewline
\begin{minipage}[t]{0.07\columnwidth}\raggedright
sd\strut
\end{minipage} & \begin{minipage}[t]{0.18\columnwidth}\raggedleft
6.62\strut
\end{minipage} & \begin{minipage}[t]{0.10\columnwidth}\raggedleft
2.49\strut
\end{minipage} & \begin{minipage}[t]{0.15\columnwidth}\raggedleft
37.85\strut
\end{minipage} & \begin{minipage}[t]{0.12\columnwidth}\raggedleft
9.68\strut
\end{minipage} & \begin{minipage}[t]{0.09\columnwidth}\raggedleft
6.75\strut
\end{minipage} & \begin{minipage}[t]{0.11\columnwidth}\raggedleft
5.47\strut
\end{minipage}\tabularnewline
\begin{minipage}[t]{0.07\columnwidth}\raggedright
range\strut
\end{minipage} & \begin{minipage}[t]{0.18\columnwidth}\raggedleft
50.22\strut
\end{minipage} & \begin{minipage}[t]{0.10\columnwidth}\raggedleft
15.01\strut
\end{minipage} & \begin{minipage}[t]{0.15\columnwidth}\raggedleft
247.39\strut
\end{minipage} & \begin{minipage}[t]{0.12\columnwidth}\raggedleft
57.54\strut
\end{minipage} & \begin{minipage}[t]{0.09\columnwidth}\raggedleft
42.81\strut
\end{minipage} & \begin{minipage}[t]{0.11\columnwidth}\raggedleft
26.93\strut
\end{minipage}\tabularnewline
\bottomrule
\end{longtable}

\hypertarget{model-specifications}{%
\section{Model Specifications}\label{model-specifications}}

\hypertarget{main-model}{%
\subsection{1. Main model}\label{main-model}}

\begin{tabular}{l c c}
\hline
 & \multicolumn{2}{c}{Unemployment rate} \\
\cline{2-3}
 & (7)F & (8)F \\
\hline
intercept              & $4.46$       & $5.28$       \\
                       & $(3.36)$     & $(2.78)$     \\
inflation              & $0.03$       &              \\
                       & $(0.07)$     &              \\
1/inflation            & $1.29$       & $1.26$       \\
                       & $(0.65)$     & $(0.64)$     \\
GDP Growth ($<$0\%)    & $-0.85^{**}$ & $-0.86^{**}$ \\
                       & $(0.26)$     & $(0.26)$     \\
GDP Growth ($>$=0\%)   & $0.19^{*}$   & $0.17^{*}$   \\
                       & $(0.08)$     & $(0.08)$     \\
Broad Money ($<$37\%)  & $-0.03^{*}$  & $-0.03^{**}$ \\
                       & $(0.01)$     & $(0.01)$     \\
Broad Money ($>$=37\%) &              & $0.01$       \\
                       &              & $(0.65)$     \\
\hline
R$^2$                  & $0.15$       & $0.15$       \\
Adj. R$^2$             & $0.11$       & $0.11$       \\
Num. obs.              & $118$        & $118$        \\
RMSE                   & $5.16$       & $5.17$       \\
\hline
\multicolumn{3}{l}{\scriptsize{$^{***}p<0.001$; $^{**}p<0.01$; $^{*}p<0.05$}}
\end{tabular}

\hypertarget{parameter-stability}{%
\subsection{2. Parameter Stability}\label{parameter-stability}}

\hypertarget{effect-of-influential-observations}{%
\subsection{3. Effect of Influential
Observations}\label{effect-of-influential-observations}}

\hypertarget{findings}{%
\section{Findings}\label{findings}}

generalization, external validity how close it is to causality, what
other confounders there may be

\hypertarget{summary}{%
\section{Summary}\label{summary}}

\hypertarget{appendix}{%
\section{Appendix}\label{appendix}}

\hypertarget{i.-model-specification}{%
\subsection{I. Model Specification}\label{i.-model-specification}}

\begin{enumerate}
\def\labelenumi{\arabic{enumi}.}
\tightlist
\item
  Investigate pattern of association for 2017 (LOESS charts)
  \textbackslash begin\{figure\}
\end{enumerate}

\{\centering \includegraphics{unemployment_analysis_report_files/figure-latex/fig17-1}

\}

\caption{Patterns of associations between pairs of variables (2017)}\label{fig:fig17}

\textbackslash end\{figure\}

\begin{figure}

{\centering \includegraphics{unemployment_analysis_report_files/figure-latex/fig_corr_17-1} 

}

\caption{Corraltion matrix of variables (2017)}\label{fig:fig_corr_17}
\end{figure}

\begin{enumerate}
\def\labelenumi{\arabic{enumi}.}
\setcounter{enumi}{1}
\tightlist
\item
  Model Comparison 2017
\end{enumerate}

\begin{figure}

{\centering \includegraphics{unemployment_analysis_report_files/figure-latex/fig_regs_17-1} 

}

\caption{Modelling unemployment with inflation and 1/inflation (2017)}\label{fig:fig_regs_17}
\end{figure}

\begin{tabular}{l c c c c c c c c}
\hline
 & \multicolumn{8}{c}{Unemployment rate} \\
\cline{2-9}
 & (1) & (2) & (3) & (4) & (5) & (6) & (7)F & (8)F \\
\hline
intercept              & $6.97^{***}$ & $7.30^{***}$ & $10.59^{***}$ & $6.51^{**}$  & $1.79$       & $2.95$       & $4.46$       & $5.28$       \\
                       & $(0.71)$     & $(0.60)$     & $(1.43)$      & $(2.10)$     & $(3.23)$     & $(3.78)$     & $(3.36)$     & $(2.78)$     \\
inflation              & $0.03$       &              & $-0.00$       & $0.06$       & $0.08$       & $0.05$       & $0.03$       &              \\
                       & $(0.08)$     &              & $(0.07)$      & $(0.07)$     & $(0.07)$     & $(0.08)$     & $(0.07)$     &              \\
1/inflation            &              & $-0.27$      &               &              &              &              &              & $0.01$       \\
                       &              & $(0.60)$     &               &              &              &              &              & $(0.65)$     \\
GDP Growth ($<$0\%)    &              &              & $1.17$        & $2.05^{*}$   & $2.04^{*}$   & $1.19$       & $1.29$       & $1.26$       \\
                       &              &              & $(0.74)$      & $(0.96)$     & $(0.88)$     & $(0.64)$     & $(0.65)$     & $(0.64)$     \\
GDP Growth ($>$=0\%)   &              &              & $-0.87^{**}$  & $-0.69^{**}$ & $-0.70^{**}$ & $-0.82^{**}$ & $-0.85^{**}$ & $-0.86^{**}$ \\
                       &              &              & $(0.27)$      & $(0.25)$     & $(0.25)$     & $(0.27)$     & $(0.26)$     & $(0.26)$     \\
Gov. Exp. ($<$22\%)    &              &              &               & $0.20$       & $0.16$       &              &              &              \\
                       &              &              &               & $(0.11)$     & $(0.12)$     &              &              &              \\
Gov. Exp. ($>$=22\%)   &              &              &               & $0.19$       & $0.17$       &              &              &              \\
                       &              &              &               & $(0.29)$     & $(0.28)$     &              &              &              \\
Broad Money ($<$37\%)  &              &              &               &              & $0.17$       & $0.19^{*}$   & $0.19^{*}$   & $0.17^{*}$   \\
                       &              &              &               &              & $(0.09)$     & $(0.09)$     & $(0.08)$     & $(0.08)$     \\
Broad Money ($>$=37\%) &              &              &               &              & $-0.02^{*}$  & $-0.02$      & $-0.03^{*}$  & $-0.03^{**}$ \\
                       &              &              &               &              & $(0.01)$     & $(0.01)$     & $(0.01)$     & $(0.01)$     \\
Savings ($<$15\%)      &              &              &               &              &              & $0.14$       &              &              \\
                       &              &              &               &              &              & $(0.14)$     &              &              \\
Savings ($>$=15\%)     &              &              &               &              &              & $-0.09$      &              &              \\
                       &              &              &               &              &              & $(0.06)$     &              &              \\
\hline
R$^2$                  & $0.00$       & $0.00$       & $0.11$        & $0.14$       & $0.17$       & $0.16$       & $0.15$       & $0.15$       \\
Adj. R$^2$             & $-0.01$      & $-0.01$      & $0.09$        & $0.10$       & $0.12$       & $0.11$       & $0.11$       & $0.11$       \\
Num. obs.              & $118$        & $118$        & $118$         & $118$        & $118$        & $118$        & $118$        & $118$        \\
RMSE                   & $5.49$       & $5.49$       & $5.23$        & $5.18$       & $5.14$       & $5.17$       & $5.16$       & $5.17$       \\
\hline
\multicolumn{9}{l}{\scriptsize{$^{***}p<0.001$; $^{**}p<0.01$; $^{*}p<0.05$}}
\end{tabular}

\hypertarget{ii.-robustness-check}{%
\subsection{II. Robustness Check}\label{ii.-robustness-check}}

\begin{enumerate}
\def\labelenumi{\arabic{enumi}.}
\item
  Dropping observations that have government expenditure above 30\%
\item
  Dropping observations that have inflation above 30\%
\end{enumerate}

\hypertarget{iii.-checking-external-validity}{%
\subsection{III. Checking External
Validity}\label{iii.-checking-external-validity}}

\begin{enumerate}
\def\labelenumi{\arabic{enumi}.}
\tightlist
\item
  Investigate pattern of association for 2016 (LOESS charts)
\end{enumerate}

\begin{center}\includegraphics{unemployment_analysis_report_files/figure-latex/fig16-1} \end{center}

\begin{enumerate}
\def\labelenumi{\arabic{enumi}.}
\setcounter{enumi}{1}
\item
  Model Comparison 2016
\item
  Investigate pattern of association for 2018 (LOESS charts)
\end{enumerate}

\begin{center}\includegraphics{unemployment_analysis_report_files/figure-latex/fig18-1} \end{center}

4.Model Comparison 2018

\end{document}
